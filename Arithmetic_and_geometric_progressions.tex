Prove that $\sum^N_1 n = \frac{1}{2}N(N+1)$.
\newline
Evaluate:
\begin{enumerate}
\item the sum of the odd integers from 11 to 99 inclusive.
\item $\sum_{n=1}^5(3n+2)$
\item $\sum^N_{n=0}(an+b)$ ($a$ and $b$ are constants)
\item $\sum_{r=0}^{10}2^r$
\item $\sum_{n=0}^N ar^{2n}$ ($a$ and $r$ are constants).
\end{enumerate}
\begin{proof}
We have
\[
\sum^N_{n=1} n = 1 + 2 + 3 + \hdots + N.
\]
This is a sum of $N$ terms. The mean of these terms, since they are consecutive numbers, is the middle term. That is half the $(N+1)^{\text{th}}$ term. So we have a sum of $N$ terms, which have a mean of $\frac{1}{2}(N+1)$. Hence their sum is the product of the mean and the number of terms:
\[
\frac{1}{2}N(N+1).
\]
\end{proof}
\begin{enumerate}
\item
\[
\sum_{n=0}^{44}11 + 2n = 45(55) = 2475.
\]
\item
\[
\sum^{5}_{n=1} (3n+2) = 5 + 8 + 11 + 14 + 17 + 20 = 75.
\]
\item
\[
\sum^{N}_{n=0}(an+b) = \frac{1}{2}(aN + 2b)(N).
\]
\item
\[
\sum_{r=0}^{10}2^r = 2^{11} - 1 = 2047.
\]
\item
\[
\sum^N_{n=0}ar^{2n} = \frac{a(r^{2N} - 1)}{r^2-1}.
\]
\end{enumerate}