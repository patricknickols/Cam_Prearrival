Write $\sqrt{3}\sin\theta + \cos\theta$ in the form $A\sin(\theta + \alpha)$, where $A$ and $\alpha$ are to be determine.
\newline
Working backwards, we see
\[
A\sin(\theta + \alpha) = A(\sin\theta\cos\alpha + \sin\alpha\cos\theta).
\]
From here we equate coefficients and see that
\[
A\cos\alpha = \sqrt{3}
\]
\[
A \sin \alpha = 1.
\]
Dividing these two equations by each other yields
\[
\tan \alpha = \frac{1}{\sqrt{3}}
\]
and so one solution is 
\[
\alpha = \frac{\pi}{6}.
\]
From there we can substitute this into either of the equations to obtain
\[
A = 2.
\]
So
\[
\sqrt{3}\sin\theta + \cos\theta = 2\sin\left(\theta + \frac{\pi}{6}\right).
\]